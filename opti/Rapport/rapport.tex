\documentclass{article}
\usepackage[T1]{fontenc}
\usepackage[utf8]{inputenc}
\usepackage[french]{babel}
\usepackage{enumitem}
\usepackage{amsmath}
\usepackage{amssymb}
\usepackage{listings}
\usepackage{graphicx}
\usepackage{hyperref}
\hypersetup{
    colorlinks,
    citecolor=black,
    filecolor=black,
    linkcolor=black,
    urlcolor=black
}

\title{Rapport du porjet d'Optimisation\newline
    \textit{One Pizza is all you need}}
\author{LOI Léo}

\begin{document}
\maketitle

\tableofcontents
\newpage

\section{Le problème :}

Nous ouvrons une pizzeria qui n'a au menu qu'une seule pizza.\\
Un client viendra dans notre pizzeria uniquement si les deux conditions suivantes sont remplies :
\begin{enumerate}
    \item Tous les ingrédients qu'il aime sont sur la pizza
    \item Aucun des ingrédients qu'il n'aime pas se trouve sur la pizza
\end{enumerate}
Nous devons décider des ingrédients qui iront sur cette pizza afin de maximiser le nombre de clients qui achèteront cette pizza.

\section{Petites instances du problème : recherche explicite}

Tout d'abord, nous pouvons nous demander ce qu'est une solution au problème.\newline
Nous avons choisit, ici, qu'une solution serait représentée par une liste d'ingredients.
Un ingrédient de cette liste est un ingrédient qu'un client à dit aimer ou détester et chaque ingrédient n'apparaît qu'au plus une fois dans la liste.

Ainsi, si nous supposons N le nombre d'ingredients disponibles au total, la liste solution aura entre 0 et N ingrédients.\newline
Mais si nous cherchons à calculer le nombre de solutions totales, la chose se complique un peu.

Soit n le nombre d'ingrédients et k la taille de la liste solution souhaitée, le nombre de combinaisons possible est calculé par la formule suivante :
$$C_{n}^{k}=\frac{n!}{k!(n-k)!}$$
Sauf qu'ici, nous cherchons le nombre de solutions totales, ce qui revient à faire le calcul précédent pour toutes les tailles de listes allant de 0 à N.\newline
On obtient ainsi le calcul suivant:
$$\sum_{k=0}^{n}\bigl(C_{n}^{k}\bigr)$$
Par exemple, si nous avons 6 ingrédients, le calcul devient :
\begin{align*}
    \sum_{k=0}^{6}\bigl(C_{6}^{k}\bigr)&=\sum_{k=0}^{6}\bigl(\frac{6!}{k!(6-k)!}\bigr)\\
    &=\frac{6!}{0!(6-0)!}+\frac{6!}{1!(6-1)!}+...+\frac{6!}{5!(6-5)!}+\frac{6!}{6!(6-6)!}\\
    &= 1 + 6 + 15 + 20 + 15 + 6 + 1\\
    &=64
\end{align*}

Nous pouvons, naïvement, essayer de produire un programme permettant de trouver la solution au problème.\newline
Nous pouvons par exemple créer le code suivant : 
\lstinputlisting[linerange={22-40}]{branch.py}


\end{document}