\documentclass{article}
\usepackage[T1]{fontenc}
\usepackage[utf8]{inputenc}
\usepackage[french]{babel}
\usepackage{enumitem}
\usepackage{amsmath}
\usepackage{amssymb}
\usepackage{listings}
\usepackage{graphicx}
\usepackage{hyperref}
\hypersetup{
    colorlinks,
    citecolor=black,
    filecolor=black,
    linkcolor=black,
    urlcolor=black
}

\title{Rapport du porjet d'Optimisation\newline
    \textit{One Pizza is all you need}}
\author{BASUALDO Lautaro et LOI Léo}

\begin{document}
\maketitle

\tableofcontents
\newpage

\section{Le problème :}

Nous ouvrons une pizzeria qui n'a au menu qu'une seule pizza.\\
Un client viendra dans notre pizzeria uniquement si les deux conditions suivantes sont remplies :
\begin{enumerate}
    \item Tous les ingrédients qu'il aime sont sur la pizza
    \item Aucun des ingrédients qu'il n'aime pas se trouve sur la pizza
\end{enumerate}
Nous devons décider des ingrédients qui iront sur cette pizza afin de maximiser le nombre de clients qui achèteront cette pizza.

\section{Petites instances du problème : recherche explicite}

Tout d'abord, nous pouvons nous demander ce qu'est une solution au problème.\newline
Nous avons choisit, ici, qu'une solution serait représentée par une liste d'ingredients.
Un ingrédient de cette liste est un ingrédient qu'un client à dit aimer ou détester et chaque ingrédient n'apparaît qu'au plus une fois dans la liste.

Ainsi, si nous supposons N le nombre d'ingredients disponibles au total, la liste solution aura entre 0 et N ingrédients.\newline
Mais si nous cherchons à calculer le nombre de solutions totales, la chose se complique un peu.

Soit n le nombre d'ingrédients et k la taille de la liste solution souhaitée, le nombre de combinaisons possible est calculé par la formule suivante :
$$C_{n}^{k}=\frac{n!}{k!(n-k)!}$$
Sauf qu'ici, nous cherchons le nombre de solutions totales, ce qui revient à faire le calcul précédent pour toutes les tailles de listes allant de 0 à N.\newline
On obtient ainsi le calcul suivant:
$$\sum_{k=0}^{n}\bigl(C_{n}^{k}\bigr)$$
Par exemple, si nous avons 6 ingrédients, le calcul devient :
\begin{align*}
    \sum_{k=0}^{6}\bigl(C_{6}^{k}\bigr)&=\sum_{k=0}^{6}\bigl(\frac{6!}{k!(6-k)!}\bigr)\\
    &=\frac{6!}{0!(6-0)!}+\frac{6!}{1!(6-1)!}+...+\frac{6!}{5!(6-5)!}+\frac{6!}{6!(6-6)!}\\
    &= 1 + 6 + 15 + 20 + 15 + 6 + 1\\
    &=64
\end{align*}

Nous pouvons, naïvement, essayer de produire un programme permettant de trouver la solution au problème.\newline
Nous pouvons par exemple créer le code suivant : 
\lstinputlisting[linerange={22-40}]{branch.py}

\section{Grosses instances du problème : métaheuristiques}
\subsection{Recuit Simulé : fait par Léo}
    Pour notre recuit simulé, entre chaque solution, nous autorisons trois mouvements différents. 
    \begin{itemize}
        \item l'ajout d'un ingrédient qui n'est pas encore dans la solution
        \item l'échange d'un ingrédient par un autre qui n'est pas encore dans la solution
        \item la suppression d'un ingrédient de la solution
    \end{itemize}
    \textit{Dû à certains problèmes, je n'ai pas pû implémenter la suite de l'algorithme. La suite du rapport concerne ce que j'aurais souhaité implémenter}\newline
    Comme expliqué dans le cour, je comptais faire une 100aine de modifications aléatoires puis de faire la moyenne de variation dans leurs scores.
    Je comptais faire débuter la température $T_0$ à 1000 puis de la faire diminuer de 0,9*$T_n$ à chaque fois que soit 120 solutions sont acceptés soit que 1000 solutions ont été tentées
    J'ai décidé d'arrêter l'algorithme si nous ne constatons plus de changement après que la température ait changé 2 fois.
    Pour cela, nous avons mis un compteur qui se réinitialise à chaque modification de solution qui détermine l'arrêt des modifications et donc de l'algorithme

\subsection{Algorithme Genetique : fait par Lautaro}
    Pour notre algorithme genetique, nous devons mettre en place differents elements propres à cet algorithme:
    \begin{itemize}
        \item La taille de la population: pour la taille de la population j'ai choisi un nombre assez grand mais je ne voulais pas qu'il soit trop élevé donc j'ai choisi 100.
        \item l'operateur de croisement: pour faire le crossover on utilise un valeur x generee aleatoirement (sous certaines contraintes)
            et on prendra la partie gauche du premier parent jusqu'a l'indice x et concatene a cela la partie apres l'indice x du deuxieme parent.
            Petite precision: je garde la meilleure recette d'une generation vers la suivante a l'indice 0 pour la conserver.
        \item la mutation: pour realiser la mutation, nous remplaçons un ingredient de la recette en mutation par un ingredient de la liste d'ingredients du probleme traité.
            J'ai choisi un probabilité de mutation basse pour commencer qui sera de 1\%.
        \item le critere d'arret: au debut mon critere d'arret etait la trouvaille d'une recette parfaite ayant tous les ingredients aimés et aucun detesté,
            mais j'ai rapidement compris que ce n'etait pas si simple. J'ai donc changé d'idée et
            j'ai choisi deux criteres d'arret qui sont: un nombre maximum de generations (qui est de 1000)
            et un arret dans le cas ou le score de la meilleure recette de chaque generation n'augmente pas pendant 15 generations d'affilé.
            J'ai ensuite testé mon algorithme sur les problemes mais quand j'ai fait le D, cela prenait trop de temps alors j'ai decidé de mettre le maximum de generations à 100 pour le probleme D.
            Finalement pour le probleme E j'ai mis le maximum de generations a 10 pour avoir un resultat car je craignais que cela prenne bien trop de temps avec une valeur plus grande.
    \end{itemize}
    Les solutions pour chaque probleme sont nommees sous la forme "solutionA\_genetique.txt" par exemple, pour chaque probleme.


\end{document}